\documentclass{article}
%\documentclass{class} Команда для начала документа, может
%содержать 6 разных классов для задания вида документа: article,letter,report,book,proc,slides
\usepackage[utf8x]{inputenc} %Данные две строчки добавляют поддержку
\usepackage[russian]{babel} %кириллицы
\author{Носков Роман, Пасютин Александр, Панчишин Даниил}%Данная команда обозначает авторов документа
\date{\today}%Данная команда обозначает дату написания документа, а today ставит сегодняшнюю дату
\title{Создание документа в \LaTeX}%Задаёт название заголовка
\begin{document}
\maketitle%Создаёт заголовок
%При создании документа обязательно должен быть \begin{document} и  \end{document}
%В них уже производится написание основного текста.
\section{Команды}
Команда section создаёт раздел, для этого ещё могут служить команды: part, section, paragraph, 
subsection, subparagraph, subsubsection
%Чтобы вынести текст на следующий строку нужно создать две пустые строки


\TeX, \LaTeX, \LaTeXe  - Команды TeX, LaTeX, LaTeXe создают данные логотипы 
%/TeX %/LaTeX, %LaTeXe

\ldots - команда ldots добавляет несколько точек

Подстрочное примечание\footnote{Пример подстрочного примечания} можно сделать при помощи команды footnote


\indent Команда indent создаёт отступ,


\noindent а noindent наоброт убирает отступ.


\emph{Для создания курсивной строки используется команда emph}

\begin{center}
	команда begin{center} делает текст по середине, 
	
	так же для конца нужна команда end{center}
\end{center}

\begin{huge}
	Так же при помощи команд begin и end можно создавать разные стили текста, которые отображаются при написании этих команд.
\end{huge}


\textmd{textmd позволяет добавить среднюю жирность тексту, а}


\textbf{textbf сделать текст жирным}

\end{document}