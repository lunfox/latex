\documentclass{article}
\usepackage[russian]{babel}
\usepackage{amssymb,amsmath}
\usepackage{pgfplots}
\pgfplotsset{compat=1.9}

\begin{document}
	$\sum_{i=1}^n n^2=\frac{n(n+1)(2n+1)(\sqrt{x^3})}{\left| |x+1|-|x-1|\right|}$
\vspace{\baselineskip}
	
	
	$P\left(A=2\middle|\frac{A^2}{B}>4\right)$

\vspace{\baselineskip}
	
	$\varlimsup_{n\to\infty}
	a_n=\inf_n\sup_{m\ge n}a_m$
\vspace{\baselineskip}
	
	$\mathcal F_x=
	\varinjlim_{U\ni x}\mathcal F(U)$
\vspace{\baselineskip}

	$\boxed{\int\limits_a^b\frac12
	(1+x)^{-3/2}dx=
	\left.-\frac{1}{\sqrt{1+x}}
	\right|_a^b}$
\vspace{\baselineskip}

	$\boxed{
		\iint_{\mathbb R^2}
		e^{-(x^2+y^2)}\,dx\,dy=\pi
	}$
\vspace{\baselineskip}

	$\begin{pmatrix}
		a_{11}-\lambda & a_{12}&a_{13}\\
		a_{21}& a_{22}-\lambda &a_{23}\\
		a_{31}& a_{32}&a_{33}-\lambda
	\end{pmatrix}$
\vspace{\baselineskip}
	
	\[
	\setcounter{MaxMatrixCols}{20}
	\begin{matrix}
		&&&&&& 1\\
		&&&&& 1 && 1\\
		&&&& 1 && 2 && 1\\
		&&& 1 && 3 && 3 && 1\\
II.3.65 && 1 && 4 && 6 && 4 && 1\\
		&1 && 5 && 10 && 10 && 5 && 1
	\end{matrix}
	\]
\vspace{\baselineskip}


\begin{tikzpicture}
	%Цветные квадраты
	\fill[red] (0,0) rectangle (3,3);
	\fill[blue] (3,3) rectangle (4,4);
	\fill[green] (0,3) rectangle (3,4);
	\fill[green] (3,0) rectangle (4,3);
	%Надписи снаружи
	\path (0,0) -- node[left] {$a$} (0,3) --
	node[left] {$b$} (0,4);
	\path (0,0) -- node[below=3pt] {$a$} (3,0) --
	node[below] {$b$} (4,0);
	\node[circle,fill=white,inner sep=2pt]
	%Надписи внутри
	at (1.5,1.5) {$a^2$};
	\node[circle,fill=white,inner sep=2pt]
	at (3.5,3.5) {$b^2$};
	\node[circle,fill=white,inner sep=2pt]
	at (1.5,3.5) {$a\cdot b$};
	\node[circle,fill=white,inner sep=2pt,rotate=90]
	at (3.5,1.5) {$a\cdot b$};
\end{tikzpicture}

\begin{tikzpicture}
	\draw[fill=green] (0,0) circle (1);
	\fill[inner color=yellow, outer color=red] (1,1) circle (1);
	\shade[ball color=blue] (2,2) circle (1);
\end{tikzpicture}

	\[\begin{tikzpicture}
		% Края доски:
		\draw[very thick] (-5,0) -- (5,0) -- (5,5) -- (-5,5) -- cycle;
		% Рисунок:
		\draw[ultra thick, ->>, blue] (-4,1)--(4,4);
		\draw[thick, ->, red] (-4,1)--(-4,4);
		\draw[thick, ->, red] (-4,1)--(4,1);
		\draw[dashed] (-4,4)--(4,4)--(4,1);
		%Подписи:
		\node at (-4.2,0.8) {\color{black} 0};
		\node at (4,4.2) {\color{blue} End of line};
		\node at (-4,4.2) {\color{red} End of line};
		\node at (4,0.7) {\color{red} End of line};
	\end{tikzpicture}
	\]
	
\begin{tikzpicture}
	\begin{axis}[
		title = Exponenta,
		xlabel = {$x$},
		ylabel = {$y$},
		minor tick num = 2
		]
		\addplot[blue] {e^x};
	\end{axis}
\end{tikzpicture}

\begin{tikzpicture}
	\begin{axis}[
		title = Парабола,
		xlabel = {$x$},
		ylabel = {$y$},
		minor tick num = 2
		]
		\addplot[magenta] {x^2};
	\end{axis}
\end{tikzpicture}

\begin{tikzpicture}
	\begin{axis}[
		title = Логарифм Х по основанию 2,
		xlabel = {$x$},
		ylabel = {$y$},
		minor tick num = 2
		]
		\addplot {log2(x)};
	\end{axis}
\end{tikzpicture}

\begin{tikzpicture}
	\begin{axis}[
		title = 3D model,
		view={110}{10}, 
		colormap/greenyellow,
		]
		\addplot3[surf] {-sin(x^2 + y^2)};
	\end{axis}
\end{tikzpicture}

\end{document}
