\documentclass[a4paper,12pt]{article}

%%% Работа с русским языком
\usepackage{cmap} %поиск в PDF
\usepackage{mathtext} %русские буквы в формулах
\usepackage[T2A]{fontenc} %кодировка
\usepackage[utf8]{inputenc} %кодировка исходного текста
\usepackage[english,russian]{babel} %локализация и переносы

%%% Матпакеты
\usepackage{amsmath,amsfonts,amssymb,amsthm,mathtools} %AMS
\usepackage{icomma} %"Умная запятая": $0,2$ --  число, $0, 2$ -- перечисление

%% Номера формул
%\mathtoolsset{showonlyrefs=true} %Показывать номера только у тех в формул,
%на которые есть \eqref{} в тексте
%\usepackage{legno} %нумеризация формул слева

%%Шрифты
\usepackage{euscript} %Шрифт Евклида
\usepackage{mathrsfs} %Красивый матшрифт

%%Свои команды
\DeclareMathOperator{\sgn}{\mathop{sgn}}

%%Перенос знаков в формулах (по Львовскому)
\newcommand*{\hm}[1]{#1\nobreak\discretionary{}
	{\hbox{$\mathsurround=0pt #1$}}{}}

%%%Заголовок
\author{Автор А.А.}
\title{Документ в \LaTeX{}}
\date{\today}

\begin{document}
	\maketitle
Первый абзац.

Второй абзац.
$2+2=4$
\[2+2=4\]

$2,4$, $2, 4$

бла бла бла бла бла бла бла бла бла бла бла бла бла бла бла $1+2\hm{+}3+4+5+6=21$

\begin{equation}\label{eq:mrmc}
	MR=MC
\end{equation}

\eqref{eq:mrmc} на стр. \pageref{eq:mrmc} --- условие максимизации прибыли.

\section{Работа с формулами}

\subsection{Дроби}

\[\frac{1+\dfrac{4}{2}}{6} = 0,5\] 

\subsection{Скобки}

\[ \left(2+\frac{9}{3}\right) \times 5 = 25\]

\[ [2+3] \]

\[ \{2+3\}\]

\subsection{Стандартные функции}

$\sin x = 5$
$\ln x = 5$
$\sgn x = 5$

\subsection{Символы}

$2 \times 2 \ne 5$

$A \cap B$, $A \cup B$

\subsection{Диакратические знаки}

$\bar x = 5$, $\tilde x = 5$

\subsection{Буквы других алфавитов}

$\tan \alpha = 1$
$\epsilon, \phi$
$\varepsilon, \varphi$

\subsection{Формулы в несколько строк}

\begin{multline}
1+2+3+4+4+4+4+4+4+4+1+2+3+4+4+4+4+4+4+1+2+3+4+4+4+4+4+4+4+4\\+4+4+4+4+4+1+2+3+4+4+4+4+4+4+1+2+3+4+4+4+4+4
\end{multline}
\end{document}