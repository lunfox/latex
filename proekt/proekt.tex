\documentclass[russian, 14pt]{beamer}
%\documentclass[handout]{beamer} %раздаточный материал
%\documentclass[aspectratio=169]{beamer}

\hypersetup{pdfpagemode=FullScreen}

%%% Работа с русским языком
\usepackage{cmap} %поиск в PDF
\usepackage{mathtext} %русские буквы в формулах
\usepackage[T2A]{fontenc} %кодировка
\usepackage[utf8]{inputenc} %кодировка исходного текста
\usepackage[english,russian]{babel} %локализация и переносы

%%Beamer по-русски
\newtheorem{rtheorem}{Теорема}
\newtheorem{rproof}{Доказательство}
\newtheorem{rexample}{Пример}

%%% Матпакеты
\usepackage{amsmath,amsfonts,amssymb,amsthm,mathtools} %AMS
\usepackage{icomma} %"Умная запятая": $0,2$ --  число, $0, 2$ -- перечисление

%% Номера формул
%\mathtoolsset{showonlyrefs=true} %Показывать номера только у тех в формул,
%на которые есть \eqref{} в тексте
%\usepackage{legno} %нумеризация формул слева

%%Свои команды
\DeclareMathOperator{\sgn}{\mathop{sgn}}

%%Перенос знаков в формулах (по Львовскому)
\newcommand*{\hm}[1]{#1\nobreak\discretionary{}
	{\hbox{$\mathsurround=0pt #1$}}{}}

%%%Работа с картинками
\usepackage{graphics} %Для вставки рисунков
\graphicspath{{images/}} %папки с картинками
\setlength\fboxsep{3pt} %отступ рамки \fbox{} от рисунка 
\setlength\fboxrule{1pt} %толщина линий рамки \fbox{}
\usepackage{wrapfig} %обтекание рисунков и таблиц текстом

%%%Работа с таблицами
\usepackage{array,tabularx,tabulary,booktabs} %дополнительная работа с таблицами
\usepackage{longtable} %длинные таблицы
\usepackage{multirow} %слияние строк в таблице

%%%Работа с видео и аудио
\usepackage{multimedia}
\usepackage{media9}
\usepackage{hyperref}

%%%Листинг кода
\usepackage{listings}
\usepackage{xcolor}

%%%Цвет таблиц
\usepackage{colortbl}

\definecolor{codegreen}{rgb}{0,0.6,0}
\definecolor{codegray}{rgb}{0.5,0.5,0.5}
\definecolor{codepurple}{rgb}{0.58,0,0.82}
\definecolor{backcolour}{rgb}{0.95,0.95,0.92}
\definecolor{Mycolor1}{HTML}{00F9DE}
\definecolor{Mycolor2}{HTML}{FFCCFF}
\definecolor{Mycolor3}{HTML}{990000}

\lstdefinestyle{mystyle}{
	backgroundcolor=\color{backcolour},   
	commentstyle=\color{codegreen},
	keywordstyle=\color{magenta},
	numberstyle=\normalsize\color{codegray},
	stringstyle=\color{codepurple},
	basicstyle=\ttfamily\normalsize,
	breakatwhitespace=false,         
	breaklines=true,                 
	captionpos=b,                    
	keepspaces=true,                 
	numbers=left,                    
	numbersep=5pt,                  
	showspaces=false,                
	showstringspaces=false,
	showtabs=false,                  
	tabsize=2
}

\lstset{style=mystyle}

\usepackage{caption}

\usetheme{Rochester} %тема оформления

%beamer theme matrix
%\usecolortheme{whale}

\usecolortheme{seahorse}

\setbeamertemplate{blocks}[rounded][shadow=true]

\setbeamertemplate{navigation symbols}{} %отключение значков навигации

\newcommand{\cm}[1]{{\color{Mycolor3}\textbackslash#1}}

\usepackage{pgfplots}
\pgfplotsset{compat=1.9}

%%%Заголовок
\author[]{Панчишин Д.И.
\and Носков Р.И.
\and Пасютин А.С.
}
\title[Презентация]{Проект \textbf{\LaTeX}}
\subtitle{Выполнили студенты 1 курса, ФИТ-204:}
\date[]{}
\institute[]{\normalsize\textbf{КемГУ}}

\begin{document}
	
	\begin{frame}
		\maketitle
	\end{frame}

\section{Цели}

\begin{frame}
	\transdissolve<1>[duration=1]
	\frametitle{\insertsection}
	\begin{itemize}
		\item[\textbullet] Научиться делать документы с высококачественной версткой текста и формул.
		\item[\textbullet] Продемонстрировать группе возможности \LaTeX'а.
	\end{itemize}
\end{frame}

\section{Задачи}

\begin{frame}
	\transblindshorizontal<1>[duration=1]
	\frametitle{\insertsection}
	\begin{itemize}
		\item[\textbullet] Изучение инструментов и макропакетов \TeX’а.
		\item[\textbullet] Получение навыков верстки текста в \LaTeX'е.
		\item[\textbullet] Создание отчета по проекту в системе \LaTeX.
	\end{itemize}
\end{frame}

\section{Индивидуальные задачи}

\begin{frame}
	\transblindsvertical<1>[duration=1]
	\frametitle{\insertsection}
	\begin{enumerate}
		\item[\textbullet]<1-> \textbf{Панчишин Даниил} - Тим-лид, создание тех задания, работа в \LaTeX’е с мат. формулами, рисунками и графиками;
		\item[\textbullet]<2-> \textbf{Носков Роман} - Работа в \LaTeX’е с инструментами для верстки текста;
		\item[\textbullet]<3-> \textbf{Пасютин Александр} - Работа в \LaTeX’е с инструментами для работы с презентациями.
	\end{enumerate}
\end{frame}

\section{Календарный план}

\begin{frame} \label{tab}
	\transboxin<1>[duration=1]
	\frametitle{\insertsection}
	\small{
	\begin{tabular}{|l|p{8,5cm}|}
		\hline
		\rowcolor{Mycolor1}
		18.02 & Распределение ролей, создание удаленного репозитория, составление календарного плана \\
		\hline
		4.03 & Изучение общего теоретического материала \\
		\hline
		\rowcolor{Mycolor1}
		18.03 & Начало работы над практической частью проекта \\
		\hline
		01.04 & Изучение отдельных аспектов \LaTeX’а, распределенных по ролям \\
		\hline
		\rowcolor{Mycolor1}
		15.04 & Создание презентации в \LaTeX, которая бы демонстрировала изученные навыки \\
		\hline
		29.04 & Создание отчета в \LaTeX, который бы демонстрировал изученные навыки
		\\
		\hline
		\rowcolor{Mycolor1}
		13.05 & Презентация результатов работы над проектом \\
		\hline
		27.05-31.05 & Защита проекта \\
		\hline
	\end{tabular}
	\hyperlink{button}{\beamerbutton{Вернуться обратно}}
}
\end{frame}

\section{Используемые средства}

\begin{frame}
	\transboxout<1>[duration=1]
	\frametitle{\insertsection}
\begin{figure}[h]
	\begin{center}
		\begin{minipage}[h]{0.4\linewidth}
			\includegraphics[width=5cm,height=5cm]{tex.pdf}
			\caption*{\huge{TeXStudio}}
		\end{minipage}
		\hfill											
		\begin{minipage}[h]{0.4\linewidth}
			\includegraphics[width=5cm,height=5cm]{miktex.pdf}
			\caption*{\huge{MiKTeX}}
		\end{minipage}
	\end{center}
\end{figure}
\end{frame}

\section{Зачем использовать \TeX \: для создания презентаций?}

\begin{frame}
	\transglitter<1>[duration=1]
	\frametitle{\insertsection}
	\begin{itemize}
		\item[\textbullet] Использование материала, набранного изначально в \LaTeX'е. 
		\item[\textbullet] Много сложных формул.
		\item[\textbullet] Переносимость и доступность. 
	\end{itemize}
\end{frame}

\section{Beamer}

\begin{frame}
	\transsplithorizontalin<1>[duration=1]
	\frametitle{\insertsection}
	\begin{itemize}
		\item[\textbullet] Beamer - класс для \LaTeX'а, предназначенный для создания презентаций.
		\item[\textbullet] Позволяет настраивать внешний вид, переходы и т.п.
	\end{itemize}
	\begin{block}{}
		 \cm{documentclass}[russian, 14pt]\{beamer\}
	\end{block}
\end{frame}

\section{Титульный слайд}

\begin{frame}
	\transsplitverticalin<1>[duration=1]
	\frametitle{\insertsection}
	\begin{block}{Преамбула}
		\cm{title}[short title]\{long title\}
		
		
		\cm{subtitle}[short subtitle]\{long subtitle\}
		
		
		\cm{author}[short name]\{long name\}
		
		
		\cm{date}[short date]\{long date\}
		
		
		\cm{institute}[short name]\{long name\}
		
		
		\cm{titlepage}
	\end{block}
\end{frame}

\section{Frame'ы}

\begin{frame}
	\transsplithorizontalout<1>[duration=1]
	\frametitle{\insertsection}
	\begin{block}{Один кадр}
		\cm{begin}\{frame\}
		
		
		\% код \LaTeX'а или текст 
		
		
		\cm{end}\{frame\}
	\end{block}
\end{frame}

\usebackgroundtemplate{\includegraphics[width=\paperwidth,height=\paperheight]{background.jpg}}

\section{Оформление}

\begin{frame}
	\transwipe<1>[duration=1]
	\frametitle{\insertsection}
	\begin{block}{}
		\cm{usetheme}\{Тема\}
		
		
		\cm{usecolortheme}\{Цветовая схема\}\\ 
		\cm{usebackgroundtemplate}\{
		\cm{includegraphics}\{background.jpg\}\}
	\end{block}
	\centering\href{https://hartwork.org/beamer-theme-matrix/}{
		\colorbox{pink}{Все встроенные темы Beamer}
	}
\end{frame}

\usebackgroundtemplate{}

\section{Overlay'и}

\begin{frame}
	\transsplitverticalout<1>[duration=1]
	\frametitle{\insertsection}
		\begin{block}{Примеры overlay'ев}
			\pause \cm{pause}
			
			
			\only<4-> {\cm{only}<4->} 
			
			
			\uncover<3-> {\cm{uncover}<3->}
			
			
			\alt<5> {\cm{alt}<5>}{Сейчас не 5 кадр}
			
			
			\temporal<6>{Сейчас не 6 кадр}{\cm{temporal}<6>}{6 кадр уже прошел}
			
			
			\begin{itemize}
				\item<7> {\cm{item}<7>} 
			\end{itemize}
		\end{block}
\end{frame}

\section{Блоки}

\begin{frame}
	\transwipe<1>[duration=1]
	\frametitle{\insertsection}
	\begin{block}{Название блока}
		Содержимое блока
	\end{block}

	\begin{block}{}
		\cm{begin}\{block\}\{Название блока\}
		
		
		Содержимое блока
		
		
		\cm{end}\{block\}
	\end{block}

\end{frame}

\section{Таблицы}

\begin{frame} \label{button}
	\transwipe<1>[duration=1]
	\frametitle{\insertsection}
	\begin{block}{}
		\begin{flushleft}
			\cm{begin}\{tabular\}\{|l|p{8,5cm}|\}
			
			
			\cm{hline}
			
			
			18.02 \& Распределение ролей, создание удаленного репозитория, составление календарного плана \textbackslash\textbackslash
			
			
			\cm{hline}
			
			
			...
			
			
			\cm{end}\{tabular\}
			
			
			\hyperlink{tab}{\beamerbutton{Посмотреть таблицу}}
		\end{flushleft}
	\end{block}
	
\end{frame}

\section{Ссылки, кнопки, рисунки}

\begin{frame} 
	\transwipe<1>[duration=1]
	\frametitle{\insertsection}
	\begin{block}{}
		\cm{hyperlink}\{label\}\{\cm{beamerbutton}\{Кнопка\}\}
		
		
		\cm{hyperlink}\{label\}\{
		
		\cm{includegraphics}[scale=0.3]\{kemsu\}\}
	\end{block}
	
	
	\centering\hyperlink{video}{\includegraphics[scale=0.3]{kemsu}}
\end{frame}

\section{Видео}

\begin{frame} \label{video}
	\transwipe<1>[duration=1]
	\frametitle{\insertsection}
	\includemedia[
	activate=pageopen,
	width=320pt,height=180pt,
	addresource=kem.mp4,
	flashvars={%
		source=kem.mp4
		&loop=true}
	]{}{VPlayer.swf}
\end{frame}

\section{Листинг кода}

\begin{frame}
	\transwipe<1>[duration=1]
	\frametitle{\insertsection}
	\begin{block}{}
		\cm{lstinputlisting}[language=C++]\{programm.cpp\}
	\end{block}
	\lstinputlisting[language=C++]{programm.cpp}
\end{frame}

\section{Анимация переходов}

\begin{frame}
	\transwipe<1>[duration=1]
	\frametitle{\insertsection}
	\begin{block}{}
		\cm{transdissolve}<слайды>[параметры]
		
		
		\cm{transblindshorizontal}<слайды>[параметры]
		
		
		\cm{transblindsvertical}<слайды>[параметры]
		
		
		\cm{transboxin}<слайды>[параметры]
		
		
		\cm{transboxout}<слайды>[параметры]
	\end{block}
\end{frame}

\section{Свои команды}

\begin{frame}
	\transwipe<1>[duration=1]
	\frametitle{\insertsection}
	\begin{block}{}
		\cm{newcommand}\{\textbackslash имя\}[число]\{действия\}
	\end{block}
	\begin{block}{}
		\cm{newcommand}\{\textbackslash cm\}[1]\{\cm{color}\{Mycolor3\}
		\cm{textbackslash}\#1\}
	\end{block}
\end{frame}

\begin{frame}[allowframebreaks]{}
	\transwipe<1>[duration=1]
	\begin{block}{}
		\cm{tableofcontents} \%[allowframebreaks]
	\end{block}
	\frametitle{Содержание}
	\tableofcontents
\end{frame}

\section{Формулы}

\begin{frame}
	\transwipe<1>[duration=1]
	\frametitle{\insertsection}
	\begin{block}{}
		\$\cm{sum}\_\{i=1\}\string^n n\string^2=\cm{frac}\{n(n+1)(2n+1)(\cm{sqrt}\{x\string^3\})\}\{\cm{left}| |x+1|-|x-1|\cm{right}|\}\$
	\end{block}
	\LARGE$\sum_{i=1}^n n^2=\frac{n(n+1)(2n+1)(\sqrt{x^3})}{\left| |x+1|-|x-1|\right|}$
\end{frame}

\begin{frame}
	\transwipe<1>[duration=1]
	\frametitle{\insertsection}
	\begin{block}{}
		\$\cm{varlimsup}\_\{n\cm{to}\cm{infty}\}
		a\_n=\cm{inf}\_n\cm{sup}\_\{m\cm{ge} n\}a\_m\$
	\end{block}
		\LARGE$\varlimsup_{n\to\infty}
a_n=\inf_n\sup_{m\ge n}a_m$
\end{frame}

\begin{frame}
	\transwipe<1>[duration=1]
	\frametitle{\insertsection}
	\begin{block}{}
		\$\cm{boxed}\{
			\cm{iint}\_\{\cm{mathbb} R\string^2\}
			e\string^\{-(x\string^2+y\string^2)\}\,dx\,dy=\cm{pi}
		\}\$
	\end{block}
	\LARGE$\boxed{
		\iint_{\mathbb R^2}
		e^{-(x^2+y^2)}\,dx\,dy=\pi
	}$
\end{frame}

\section{Матрицы}

\begin{frame}
	\transwipe<1>[duration=1]
	\frametitle{\insertsection}
	\begin{block}{}
		\cm{begin}\{pmatrix\}
		
		
		...
		
		
		\cm{end}\{pmatrix\}
	\end{block}
	\LARGE$\begin{pmatrix}
		a_{11}-\lambda & a_{12}&a_{13}\\
		a_{21}& a_{22}-\lambda &a_{23}\\
		a_{31}& a_{32}&a_{33}-\lambda
	\end{pmatrix}$
\end{frame}

\begin{frame}
	\transwipe<1>[duration=1]
	\frametitle{\insertsection}
	\begin{block}{}
		\cm{setcounter}\{MaxMatrixCols\}\{20\}
		
		
		\cm{begin}\{matrix\}
		
		
		...
		
		
		\cm{end}\{matrix\}
	\end{block}
	\[
	\setcounter{MaxMatrixCols}{20}
	\begin{matrix}
		&&&&&& 1\\
		&&&&& 1 && 1\\
		&&&& 1 && 2 && 1\\
		&&& 1 && 3 && 3 && 1\\
		&& 1 && 4 && 6 && 4 && 1\\
		&1 && 5 && 10 && 10 && 5 && 1
	\end{matrix}
	\]
\end{frame}

\section{Псевдорисунки}

\begin{frame}
	\transwipe<1>[duration=1]
	\frametitle{\insertsection}
\begin{figure}[h]
	\begin{center}
		\begin{minipage}[h]{0.4\linewidth}
\begin{tikzpicture}
	%Цветные квадраты
	\fill[red] (0,0) rectangle (3,3);
	\fill[blue] (3,3) rectangle (4,4);
	\fill[green] (0,3) rectangle (3,4);
	\fill[green] (3,0) rectangle (4,3);
	%Надписи снаружи
	\path (0,0) -- node[left] {$a$} (0,3) --
	node[left] {$b$} (0,4);
	\path (0,0) -- node[below=3pt] {$a$} (3,0) --
	node[below] {$b$} (4,0);
	\node[circle,fill=white,inner sep=2pt]
	%Надписи внутри
	at (1.5,1.5) {$a^2$};
	\node[circle,fill=white,inner sep=2pt]
	at (3.5,3.5) {$b^2$};
	\node[circle,fill=white,inner sep=2pt]
	at (1.5,3.5) {$a\cdot b$};
	\node[circle,fill=white,inner sep=2pt,rotate=90]
	at (3.5,1.5) {$a\cdot b$};
\end{tikzpicture}
		\end{minipage}
		\hfill											
		\begin{minipage}[h]{0.4\linewidth}
\begin{tikzpicture}
	\draw[fill=green] (0,0) circle (1);
	\fill[inner color=yellow, outer color=red] (1,1) circle (1);
	\shade[ball color=blue] (2,2) circle (1);
\end{tikzpicture}
		\end{minipage}
	\end{center}
\end{figure}

\end{frame}

\begin{frame}
	\transwipe<1>[duration=1]
	\frametitle{\insertsection}
	\begin{block}{}
		\cm{begin}\{tikzpicture\}
		
		...
		
		\cm{end}\{tikzpicture\}
		
		\cm{fill}
		
		\cm{path}
		
		\cm{node}
		
		\cm{draw}
		
		\cm{shade}
	\end{block}
\end{frame}

\begin{frame}
	\transwipe<1>[duration=1]
	\frametitle{\insertsection}
	\[\begin{tikzpicture}
		% Края доски:
		\draw[very thick] (-5.5,0) -- (5.5,0) -- (5.5,5.5) -- (-5.5,5.5) -- cycle;
		% Рисунок:
		\draw[ultra thick, ->>, blue] (-4,1)--(4,4);
		\draw[thick, ->, red] (-4,1)--(-4,4);
		\draw[thick, ->, red] (-4,1)--(4,1);
		\draw[dashed] (-4,4)--(4,4)--(4,1);
		%Подписи:
		\node at (-4.2,0.8) {\color{black} 0};
		\node at (4,4.2) {\color{blue} End of line};
		\node at (-4,4.2) {\color{red} End of line};
		\node at (4,0.7) {\color{red} End of line};
	\end{tikzpicture}
	\]
\end{frame}

\begin{frame}
	\transwipe<1>[duration=1]
	\frametitle{\insertsection}
	\includegraphics[scale=0.6]{str}
\end{frame}

\section{Графики}

\begin{frame}
	\transwipe<1>[duration=1]
	\frametitle{\insertsection}
	\centering
	\begin{tikzpicture}
		\begin{axis}[
			title = Парабола,
			%xlabel = {$x$},
			%ylabel = {$y$},
			minor tick num = 2
			]
			\addplot[magenta] {x^2};
		\end{axis}
	\end{tikzpicture}
\end{frame}

\begin{frame}
	\transwipe<1>[duration=1]
	\frametitle{\insertsection}
	\centering
	\begin{tikzpicture}
		\begin{axis}[
			title = Логарифм Х по основанию 2,
			%xlabel = {$x$},
			%ylabel = {$y$},
			minor tick num = 2
			]
			\addplot {log2(x)};
		\end{axis}
	\end{tikzpicture}
\end{frame}

\begin{frame}
	\transwipe<1>[duration=1]
	\frametitle{\insertsection}
	\centering
	\begin{tikzpicture}
		\begin{axis}[
			title = 3D model,
			view={110}{10}, 
			colormap/greenyellow,
			]
			\addplot3[surf] {-sin(x^2 + y^2)};
		\end{axis}
	\end{tikzpicture}
\end{frame}

\begin{frame}
	\transwipe<1>[duration=1]
	\frametitle{\insertsection}
	\begin{block}{}
		\cm{begin}\{axis\}
		
		...
		
		\cm{end}\{axis\}
		
		\cm{addplot}[magenta] \{x\string^2\}
		
		\cm{addplot}\{log2(x)\}
		
		\cm{addplot}3[surf] \{-sin(x\string^2 + y\string^2)\};
	\end{block}
\end{frame}

\section{Итоговый результат}

\begin{frame}
	\transwipe<1>[duration=1]
	\frametitle{\insertsection}
\end{frame}

\begin{frame}
	\transwipe<1>[duration=1]
	\centering
	\LARGE{Спасибо за внимание!}
\end{frame}

\end{document}