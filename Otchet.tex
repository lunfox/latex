\documentclass{article}
\usepackage[utf8x]{inputenc}
\usepackage[russian]{babel}
\usepackage{amsmath}
\author{Носков Роман, Пасютин Александр, Панчишин Даниил}
\title{Инструменты для оформления научных статей и презентаций (верстка текстового документа в Latex, оформление элементов текстового документа в \LaTeX, презентации в \LaTeX, работа с видео в \LaTeX-презентациях).}
\begin{document}
	\begin{center}
		\scriptsize{МИНИСТЕРСТВО НАУКИ И ВЫСШЕГО ОБРАЗОВАНИЯ РОССИЙСКОЙ ФЕДЕРАЦИИ
		
		Федеральное государственное бюджетное образовательное учреждение высшего образования
		
		«Кемеровский государственный университет»
		
		Институт фундаментальных наук
		
		Кафедра ЮНЕСКО по информационным вычислительным технологиям
	}
	\vspace{\baselineskip}
		
			\LARGE{\textbf{ОТЧЁТ}}
		
		\normalsize по учебной практике, технологической (проектно-технологической) практике
		
		проект «Инструменты для оформления научных статей и презентаций (верстка текстового документа в LaTeX, оформление элементов текстового документа в LaTeX, презентации в LaTeX, работа с видео в LaTeX-презентациях»
	\end{center}
\textbf{Выполнили:}

\noindent студенты направления подготовки 02.03.02 Фундаментальная информатика и информационные технологии, направленности (профиля) подготовки «Информатика и компьютерные науки»

\begin{flushright}
	
$\underset{\text{(ФИО)}}{\underline{\hspace{0.3\textwidth}}}$ $\underset{\text{(Оценка)}}{\underline{\hspace{0.1\textwidth}}}$ 
\vspace{\baselineskip}
		
$\underset{\text{(ФИО)}}{\underline{\hspace{0.3\textwidth}}}$ $\underset{\text{(Оценка)}}{\underline{\hspace{0.1\textwidth}}}$ 
\vspace{\baselineskip}

$\underset{\text{(ФИО)}}{\underline{\hspace{0.3\textwidth}}}$ $\underset{\text{(Оценка)}}{\underline{\hspace{0.1\textwidth}}}$ 
\end{flushright}
\newpage
\section{Оглавление}
1. Описание проекта





	\section*{Актуальность, теоретическая и практическая значимость:}
	
	
	\noindent\emph{Актуальность:}
	
	
	\noindent Нам, как студентам, необходимо знание редактора для оформления научных статей и презентаций.\\
	
	
	\noindent\emph{Теоретическая значимость:}
	
	
	\noindent Умение создания разверстанных научных статей и презентаций.\\
	
	
	\noindent\emph{Практическая значимость:}
	
	
	\noindent Наш проект позволит продемонстрировать группе возможности LaTeX’а.
	
	
	\section*{Цели и задачи:}
	
	
	\emph{Цель данной работы:} научиться делать документы с высококачественной версткой текста, формул и других объектов.\\
	
	
	\noindent\emph{Задачи:} изучение инструментов и макропакетов ТеХ'а, получение навыков верстки текста в ЛаТеХ'е, создание отчета по проекту в системе ЛаТеХ.
	
	
	\section*{Конечный результат:}
	Презентация и отчет о \LaTeX, созданные в \LaTeX.
	
	\section{Ход работы}
	3.03 – 18.03:
	
	Изучили особенности форматирования текста в системе LaTeX (русификация, шрифты, стили, разделы, интервалы, переносы)
	
	Изучили набор математических формул (строчные и выключные формулы, дроби, скобки, стандартные функции, символы, диакратические знаки и буквы других алфавитов)
	
	Создание презентаций (знакомство с пакетом Beamer, темы оформления, создание титульного слайда, оглавление презентации, поочередное появление объектов, выделение информации при помощи блоков и т.д.)
	
	\noindent 18.03-15.04:
	
	Создали в PowerPoint шаблон презентации, словарь команд \LaTeX для изменения шрифтов, заготовки отчета и презентации в \LaTeX
	
\end{document}