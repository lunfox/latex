\documentclass{article}
\usepackage[utf8x]{inputenc}
\usepackage[russian]{babel}
\author{Носков Роман, Пасютин Александр, Панчишин Даниил}
\title{Инструменты для оформления научных статей и презентаций (верстка текстового документа в Latex, оформление элементов текстового документа в \LaTeX, презентации в \LaTeX, работа с видео в \LaTeX-презентациях).}
\begin{document}
	\maketitle
	\section*{Актуальность, теоретическая и практическая значимость:}
	
	
	\noindent\emph{Актуальность:}
	
	
	\noindent Нам, как студентам, необходимо знание редактора для оформления научных статей и презентаций.\\
	
	
	\noindent\emph{Теоретическая значимость:}
	
	
	\noindent Умение создания разверстанных научных статей и презентаций.\\
	
	
	\noindent\emph{Практическая значимость:}
	
	
	\noindent Наш проект позволит продемонстрировать группе возможности LaTeX’а.
	
	
	\section*{Цели и задачи:}
	
	
	\emph{Цель данной работы:} научиться делать документы с высококачественной версткой текста, формул и других объектов.\\
	
	
	\noindent\emph{Задачи:} изучение инструментов и макропакетов ТеХ'а, получение навыков верстки текста в ЛаТеХ'е, создание отчета по проекту в системе ЛаТеХ.
	
	
	\section*{Конечный результат:}
	Презентация и отчет о \LaTeX, созданные в \LaTeX.
	
	\section{Распределение по ролям:}
	
\end{document}