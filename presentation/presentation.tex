\documentclass[russian]{beamer}
%\documentclass[handout]{beamer} %раздаточный материал
%\documentclass[aspectratio=169]{beamer}

%%% Работа с русским языком
\usepackage{cmap} %поиск в PDF
\usepackage{mathtext} %русские буквы в формулах
\usepackage[T2A]{fontenc} %кодировка
\usepackage[utf8]{inputenc} %кодировка исходного текста
\usepackage[english,russian]{babel} %локализация и переносы

%%Beamer по-русски
\newtheorem{rtheorem}{Теорема}
\newtheorem{rproof}{Доказательство}
\newtheorem{rexample}{Пример}

%%% Матпакеты
\usepackage{amsmath,amsfonts,amssymb,amsthm,mathtools} %AMS
\usepackage{icomma} %"Умная запятая": $0,2$ --  число, $0, 2$ -- перечисление

%% Номера формул
%\mathtoolsset{showonlyrefs=true} %Показывать номера только у тех в формул,
%на которые есть \eqref{} в тексте
%\usepackage{legno} %нумеризация формул слева

%%Свои команды
\DeclareMathOperator{\sgn}{\mathop{sgn}}

%%Перенос знаков в формулах (по Львовскому)
\newcommand*{\hm}[1]{#1\nobreak\discretionary{}
	{\hbox{$\mathsurround=0pt #1$}}{}}

%%%Работа с картинками
\usepackage{graphics} %Для вставки рисунков
\graphicspath{{images/}} %папки с картинками

\usetheme{Montpellier} %тема оформления
%beamer theme matrix
\usecolortheme{crane}

%%%Заголовок
\author[Автор]{Имя автора}
\title[Презентация]{Название презентации}
\subtitle{Подзаголовок}
\date[Дата]{\today}
\institute[КемГУ]{Кемеровский государственный университет}

\begin{document}
	
\begin{frame}
	\maketitle
\end{frame}

\section{Оглавление}
\begin{frame}
	\frametitle{Оглавление}
	\tableofcontents %оглавление
\end{frame}

\section{Поочередное появление объектов}
\subsection{Команда pause}

\begin{frame}
	\frametitle{\insertsection}
	\framesubtitle{\insertsubsection}
	\begin{itemize}
		\item beamer --- это \alert{удобный} \textbf{пакет} для
		создания презентаций.
		\item Первый слайд. \pause
		\item Паузу можно поставить в любом \pause месте.
		\item Режим handout для печати, убирает паузы.
	\end{itemize}
\end{frame}

\begin{frame}
	\frametitle{Поочередное появление пунктов списка}
	\framesubtitle{\insertsubsection}
	\uncover<4->{Строчка появляется не сразу, но занимает место.}
	\only<5->{Строчка появляется не сразу, но не занимает место.}
	\begin{enumerate}
		\item<1-5> Сначала появляется первый и последний пункт
		(первый потом исчезнет).
		\item<2-> Потом второй
		\item<3-> И наконец третий
		\item<1-> Последний пункт появляется вместе с первым
		\item<6-> В самом конце первый пункт исчезает, зато появляется
		картинка: ---.
	\end{enumerate}
	\alt<4>{Это \alert{четвертый} слайд.}{Это не четвертый слайд.}
	\temporal<3-4>{Слайды 1, 2}{Слайды 3, 4}{Слайды 5, 6, 7, ...}
\end{frame}

\begin{frame}{Пример} \label{lab}
	\textbf{This line is bold on all three slides.}
	\textbf<2>{This line is bold on the second slide.}
	\textbf<3>{This line is bold on the third slide.}
	\textbf<3-4>{Строчка полужирная на 3-м и 4-м слайде.}
	\color<3-4>[RGB]{255,0,0} Этот текст красный только на слайдах 3-4.
\end{frame}

\section{Выделение информации}
\subsection{Блоки}

\begin{frame}
	\frametitle{\insertsection}
	\framesubtitle{\insertsubsection}
	\begin{block}{Первый блок}
		Текст первого блока
	\end{block}
	\begin{block}{Второй блок}
		Текст второго блока
	\end{block}
	\begin{block}{Кнопка}
		\hyperlink{lab}{\beamerbutton{Тык}}
	\end{block}
\end{frame}

\begin{frame}
	\frametitle{\insertsection}
	\framesubtitle{\insertsubsection}
	\begin{rtheorem}
		Формулировка теоремы.
	\end{rtheorem}
	\begin{rproof}
		Текст доказательства.
	\end{rproof}
	\begin{rexample}
		Текст примера.
	\end{rexample}
\end{frame}
\end{document}

	