\documentclass[russian]{beamer}
%\documentclass[handout]{beamer} %раздаточный материал
%\documentclass[aspectratio=169]{beamer}

%%% Работа с русским языком
\usepackage{cmap} %поиск в PDF
\usepackage{mathtext} %русские буквы в формулах
\usepackage[T2A]{fontenc} %кодировка
\usepackage[utf8]{inputenc} %кодировка исходного текста
\usepackage[english,russian]{babel} %локализация и переносы

%%Beamer по-русски
\newtheorem{rtheorem}{Теорема}
\newtheorem{rproof}{Доказательство}
\newtheorem{rexample}{Пример}

%%% Матпакеты
\usepackage{amsmath,amsfonts,amssymb,amsthm,mathtools} %AMS
\usepackage{icomma} %"Умная запятая": $0,2$ --  число, $0, 2$ -- перечисление

%% Номера формул
%\mathtoolsset{showonlyrefs=true} %Показывать номера только у тех в формул,
%на которые есть \eqref{} в тексте
%\usepackage{legno} %нумеризация формул слева

%%Свои команды
\DeclareMathOperator{\sgn}{\mathop{sgn}}

%%Перенос знаков в формулах (по Львовскому)
\newcommand*{\hm}[1]{#1\nobreak\discretionary{}
	{\hbox{$\mathsurround=0pt #1$}}{}}

%%%Работа с картинками
\usepackage{graphics} %Для вставки рисунков
\graphicspath{{images/}} %папки с картинками
\setlength\fboxsep{3pt} %отступ рамки \fbox{} от рисунка 
\setlength\fboxrule{1pt} %толщина линий рамки \fbox{}
\usepackage{wrapfig} %обтекание рисунков и таблиц текстом

%%%Работа с таблицами
\usepackage{array,tabularx,tabulary,booktabs} %дополнительная работа с таблицами
\usepackage{longtable} %длинные таблицы
\usepackage{multirow} %слияние строк в таблице

%%%Работа с видео и аудио
\usepackage{multimedia}
\usepackage{media9}
\usepackage{hyperref}

\usetheme{Montpellier} %тема оформления
%beamer theme matrix
\usecolortheme{crane}

%%%Заголовок
\author[Автор]{Имя автора}
\title[Презентация]{Название презентации}
\subtitle{Подзаголовок}
\date[Дата]{\today}
\institute[КемГУ]{Кемеровский государственный университет}

\begin{document}
	
\begin{frame}
	\maketitle
\end{frame}

\section{Оглавление}

\begin{frame}
	\frametitle{Оглавление}
	\tableofcontents %оглавление
\end{frame}

\section{Поочередное появление объектов}

\subsection{Команда pause}

\begin{frame}
	\frametitle{\insertsection}
	\framesubtitle{\insertsubsection}
	\begin{itemize}
		\item beamer --- это \alert{удобный} \textbf{пакет} для
		создания презентаций.
		\item Первый слайд. \pause
		\item Паузу можно поставить в любом \pause месте.
		\item Режим handout для печати, убирает паузы.
	\end{itemize}
\end{frame}

\begin{frame}
	\frametitle{Поочередное появление пунктов списка}
	\framesubtitle{\insertsubsection}
	\uncover<4->{Строчка появляется не сразу, но занимает место.}
	\only<5->{Строчка появляется не сразу, но не занимает место.}
	\begin{enumerate}
		\item<1-5> Сначала появляется первый и последний пункт
		(первый потом исчезнет).
		\item<2-> Потом второй
		\item<3-> И наконец третий
		\item<1-> Последний пункт появляется вместе с первым
		\item<6-> В самом конце первый пункт исчезает, зато появляется
		картинка: ---.
	\end{enumerate}
	\alt<4>{Это \alert{четвертый} слайд.}{Это не четвертый слайд.}
	\temporal<3-4>{Слайды 1, 2}{Слайды 3, 4}{Слайды 5, 6, 7, ...}
\end{frame}

\begin{frame}{Пример} \label{lab}
	\textbf{This line is bold on all three slides.}
	\textbf<2>{This line is bold on the second slide.}
	\textbf<3>{This line is bold on the third slide.}
	\textbf<3-4>{Строчка полужирная на 3-м и 4-м слайде.}
	\color<3-4>[RGB]{255,0,0} Этот текст красный только на слайдах 3-4.
\end{frame}

\section{Выделение информации}

\subsection{Блоки}

\begin{frame}
	\frametitle{\insertsection}
	\framesubtitle{\insertsubsection}
	\begin{block}{Первый блок}
		Текст первого блока
	\end{block}
	\begin{block}{Второй блок}
		Текст второго блока
	\end{block}
	\begin{block}{Кнопка}
		\hyperlink{lab}{\beamerbutton{Тык}}
	\end{block}
\end{frame}

\begin{frame}
	\frametitle{\insertsection}
	\framesubtitle{\insertsubsection}
	\begin{rtheorem}
		Формулировка теоремы.
	\end{rtheorem}
	\begin{rproof}
		Текст доказательства.
	\end{rproof}
	\begin{rexample}
		Текст примера.
	\end{rexample}
\end{frame}

\section{Рисунки}

\subsection{Растровые}

\begin{frame}
	\includegraphics[scale=0.2]{kemsu_logo}	%масштаб
	\includegraphics[width=2cm,height=2cm]{kemsu_logo} %ширина, высота
	\includegraphics[width=2cm,height=2cm,keepaspectratio]{kemsu_logo} %сохранение пропорций
\end{frame}

\subsection{Векторные}

\begin{frame}
	\includegraphics[scale=0.2]{logo}
\end{frame}

\section{Таблицы}

\setlength{\extrarowheight}{6mm}

\begin{frame}
\begin{tabular}{|p{4cm}|c|r|}
	\hline
	 очень длинное предложение с большим набором символов & $6\times2$ & 13 \\
	 21 &  $\displaystyle\frac{x}{y}$ & 23 \\[6mm]
	\hline
	 31 & 32 & 33 \\
	\hline
\end{tabular}
\setlength{\extrarowheight}{0mm}
\end{frame}

\begin{frame}
	\begin{tabularx}{\textwidth}{|X|X|X|}
		\hline 
		Очень-очень длинное предложение из многих слов & Текст покороче & 
		Очень-очень длинное предложение из многих слов Очень-очень длинное предложение из многих слов \\
		\hline
	\end{tabularx}
\end{frame}

\begin{frame}
	\begin{table}[h]
		\begin{center}
			\caption{Название таблицы}
			\begin{tabular}{|c|c|c|c||l|c|c|r|c|c|}
				\hline
				1 & 2 & 3 & 4 & 5 & 6 & 7 & 8 & 9 & 10 \\ \hline
				Первый & Второй & \multicolumn{3}{|c|}{Третий -- пятый}
				& & & Восьмой & & \\
				\cline{1-7} \cline{9-10}
				1 & 2 & 3 & 4 & 5 & 6 & 7 & 8 & 9 & 10 \\ \hline \hline
				1 & 2 & 3 & 4 & 5 & 6 & 7 & 8 & 9 & 10 \\ \hline
				\multirow{2}{*}{Две строки} & 2 & 3 & 4 & 5 & 6 & 7 & 8 & 9 & 10 \\ \cline{2-10}
				& 2 & 3 & 4 & 5 & 6 & 7 & 8 & 9 & 10 \\  \hline
			\end{tabular}
		\end{center}
	\end{table}
\end{frame}

\section{Видео}
\subsection{С использованием флеш плеера}
\begin{frame}
	\frametitle{\insertsubsection}
	\includemedia[
	activate=pageopen,
	width=320pt,height=180pt,
	addresource=AMOGUS.mp4,
	flashvars={%
		source=AMOGUS.mp4
		&loop=true}
	]{}{VPlayer.swf}
\end{frame}

\subsection{Использование ссылки на файл}
\begin{frame}
	\begin{center}
		\frametitle{\insertsubsection}
		\href{run:AMOGUS.avi}{\includegraphics[scale=0.25]{AMOGUS.jpg}}
	\end{center}
\end{frame}
\end{document}